\documentclass[useAMS,usenatbib]{mn2e}

\voffset=-0.8in

% Packages:
\usepackage{graphicx}
\usepackage{amsmath}
\usepackage{xspace}
\usepackage{bbm}

% Bold symbols
\renewcommand{\btheta}{\boldsymbol{\theta}}
\newcommand{\bx}{\boldsymbol{x}}

% \onecolumn
%%%%%%%%%%%%%%%%%%%%%%%%%%%%%%%%%%%%%%%%%%%%%%%%%%%%%%%%%%%%%%%%%%%%%%%%%%%%%%

\title[Hierarchical Reverberation Mapping]
{Hierarchical Reverberation Mapping}
    
\author[Brewer and Elliott]{%
  Brendon~J.~Brewer$^{1}$\thanks{bj.brewer@auckland.ac.nz},
  Tom Elliott$^{1}$
  \medskip\\
  $^1$Department of Statistics, The University of Auckland, Private Bag 92019, Auckland 1142, New Zealand}

%%%%%%%%%%%%%%%%%%%%%%%%%%%%%%%%%%%%%%%%%%%%%%%%%%%%%%%%%%%%%%%%%%%%%%%%%%%%%%

\begin{document}
             
\date{To be submitted to MNRAS Letters}
             
\maketitle

\label{firstpage}

%%%%%%%%%%%%%%%%%%%%%%%%%%%%%%%%%%%%%%%%%%%%%%%%%%%%%%%%%%%%%%%%%%%%%%%%%%%%%%

\begin{abstract}

\end{abstract}

\begin{keywords}

\end{keywords}

%%%%%%%%%%%%%%%%%%%%%%%%%%%%%%%%%%%%%%%%%%%%%%%%%%%%%%%%%%%%%%%%%%%%%%%%%%%%%%

\section{Introduction}


Recently, \citet{2012MNRAS.427.2701F, 2013MNRAS.434L..16F} introduced an
innovative approach to reverberation mapping where the results from multiple
AGN can be combined to yield inferences about the entire sample of AGN,
despite the fact that the constraints on any individual AGN are poor.



\section{Combining Inferences About Multiple Objects}
Consider a sample of $N$ objects, each of which has parameters $\theta_i$.
The inference about object $i$ is described by a posterior distribution
\begin{eqnarray}
p(\theta_i | x_i) \propto \pi(\theta_i)p(x_i | \theta_i)\label{eq:individual}
\end{eqnarray}
We shall assume that this inference was carried out individually
on each object $i$, all using a common prior $\pi(\theta_i)$. Rather than
analysing all objects together, which can be very computationally expensive,
we can use the posterior distributions from Equation~\ref{eq:individual}
to reconstruct the results that would be obtained if we did analyse all
objects together.

In a hierarchical model, the posterior distribution for
$\btheta = \{\theta_1, ..., \theta_N\}$ and some hyperparameters $\alpha$
given the data 
$\bx = \{x_1, ..., x_N\}$
should actually be
\begin{eqnarray}
p(\alpha, \btheta | \bx) &\propto&
p(\alpha)p(\btheta|\alpha)p(\bx | \btheta, \alpha)\\
&=& p(\alpha)\prod_{i=1}^N f(\theta_i|\alpha)p(x_i | \theta_i)
\end{eqnarray}

\section{The Inference for a Single Object}
The posterior distribution for the lag of a single object $i$ is obtained
by fitting the following model.

\begin{eqnarray}
blah
\end{eqnarray}

Note that this is already not the cross-correlation.

\section{Demonstration on Simulated Data}


\section*{Acknowledgements}




\begin{thebibliography}{99}
\bibitem[\protect\citeauthoryear{Fine et al.}{2013}]{2013MNRAS.434L..16F} 
Fine S., et al., 2013, MNRAS, 434, L16 


\bibitem[\protect\citeauthoryear{Fine et al.}{2012}]{2012MNRAS.427.2701F} 
Fine S., et al., 2012, MNRAS, 427, 2701 

\end{thebibliography}



\end{document}

%%%%%%%%%%%%%%%%%%%%%%%%%%%%%%%%%%%%%%%%%%%%%%%%%%%%%%%%%%%%%%%%%%%%%%%%%%%%%%
