\documentclass[useAMS,usenatbib]{mn2e}

\voffset=-0.8in

% Packages:
\usepackage{graphicx}
\usepackage{amsmath}
\usepackage{xspace}
\usepackage{dsfont}

% Bold symbols
\renewcommand{\btheta}{\boldsymbol{\theta}}
\newcommand{\bx}{\boldsymbol{x}}

% \onecolumn
%%%%%%%%%%%%%%%%%%%%%%%%%%%%%%%%%%%%%%%%%%%%%%%%%%%%%%%%%%%%%%%%%%%%%%%%%%%%%%

\title[Hierarchical Reverberation Mapping]
{Hierarchical Reverberation Mapping}
    
\author[Brewer and Elliott]{%
  Brendon~J.~Brewer$^{1}$\thanks{bj.brewer@auckland.ac.nz},
  Tom M. Elliott$^{1}$
  \medskip\\
  $^1$Department of Statistics, The University of Auckland, Private Bag 92019, Auckland 1142, New Zealand}

%%%%%%%%%%%%%%%%%%%%%%%%%%%%%%%%%%%%%%%%%%%%%%%%%%%%%%%%%%%%%%%%%%%%%%%%%%%%%%

\begin{document}
             
\date{To be submitted to MNRAS Letters}
             
\maketitle

\label{firstpage}

%%%%%%%%%%%%%%%%%%%%%%%%%%%%%%%%%%%%%%%%%%%%%%%%%%%%%%%%%%%%%%%%%%%%%%%%%%%%%%

\begin{abstract}
Reverberation mapping (RM) is an important technique in studies of active
galactic nuclei (AGN). The key idea of RM is to measure the time lag $\tau$
between variations in the continuum emission from the accretion disc
and subsequent response of the broad line region (BLR). The measurement of
$\tau$ is typically used to estimate the physical size of the BLR and is
combined with other measurements to estimate the black hole mass $M_{\rm BH}$.
A major difficulty with RM campaigns is the large amount of data needed to
measure $\tau$.
\end{abstract}

\begin{keywords}

\end{keywords}

%%%%%%%%%%%%%%%%%%%%%%%%%%%%%%%%%%%%%%%%%%%%%%%%%%%%%%%%%%%%%%%%%%%%%%%%%%%%%%

\section{Introduction}


There are deeper issues at play in reverberation mapping, for instance, the
mean lag $\tau$ is not the same as the mean radius of the BLR distribution,
and the mean lag $\tau$ is also not what the common cross-correlation
techniques measures. In this letter we will ignore these complications.


Recently, \citet{2012MNRAS.427.2701F, 2013MNRAS.434L..16F} introduced an
innovative approach to reverberation mapping where the results from multiple
AGN can be combined to yield inferences about the entire sample of AGN,
despite the fact that the constraints on any individual AGN are poor. Rather
than accurately measuring $\tau$ in a single object, it is possible to roughly
measure $\tau$ for a large number of objects, and to infer properties about
the distribution of $\tau$ values in the sample of objects (and hence in
a broader population, if the sample can be considered representative).

We implement this idea with a Bayesian hierarchical model
\citep{2012arXiv1208.3036L}.
These models are becoming increasingly common in astrophysics and have been
used in a variety of different fields
\citep[e.g.][]{extreme_deconvolution, loredo, kelly}.

\section{Combining Inferences About Multiple Objects}
Consider a sample of $N$ objects, each of which has parameters $\theta_i$.
The inference about object $i$ is described by a posterior distribution
\begin{eqnarray}
p(\theta_i | x_i) \propto \pi(\theta_i)p(x_i | \theta_i)\label{eq:individual}
\end{eqnarray}
We shall assume that this inference was carried out individually
on each object $i$, all using a common prior $\pi(\theta_i)$. Rather than
analysing all objects together, which can be very computationally expensive,
we can use the posterior distributions from Equation~\ref{eq:individual}
to reconstruct the results that would be obtained if we did analyse all
objects together.

In a hierarchical model, the posterior distribution for
$\btheta = \{\theta_1, ..., \theta_N\}$ and some hyperparameters $\alpha$
given the data 
$\bx = \{x_1, ..., x_N\}$
should actually be
\begin{eqnarray}
p(\alpha, \btheta | \bx) &\propto&
p(\alpha)p(\btheta|\alpha)p(\bx | \btheta, \alpha)\\
&=& p(\alpha)\prod_{i=1}^N f(\theta_i|\alpha)p(x_i | \theta_i)
\end{eqnarray}
The marginal posterior distribution for the hyperparameters $\alpha$ is

\begin{eqnarray}
p(\alpha | \bx) &=&
\int p(\alpha, \btheta|\bx) \, d\btheta\\
&\propto& p(\alpha)\int \prod_{i=1}^N f(\theta_i|\alpha)p(x_i | \theta_i) d^N\theta\\
&\propto& p(\alpha) \prod_{i=1}^N \int f(\theta_i|\alpha)p(x_i | \theta_i) d\theta_i\\
&\propto& p(\alpha) \prod_{i=1}^N \int \frac{f(\theta_i|\alpha)}{\pi(\theta)}p(x_i | \theta_i) \pi(\theta_i)d\theta_i\\
&\propto& p(\alpha) \prod_{i=1}^N \mathds{E}\left[\frac{f(\theta_i|\alpha)}{\pi(\theta)}p(x_i | \theta_i)\right]
\end{eqnarray}
where the expectation is taken with respect to the individual object posterior
of Equation~\ref{eq:individual}. This result enables us to reconstruct the
posterior distribution for the hyperparameters even though the individual object
inferences were made without the hierarchical structure in the prior.



\section{The Inference for a Single Object}
The posterior distribution for the lag of a single object $i$ is obtained
by fitting the following model.

\begin{eqnarray}
blah
\end{eqnarray}

Note that this is already not the cross-correlation.

The prior for the underlying time variation of the continuum emission is
a continuous autoregressive process of order 1, or a CAR(1) model. These models
have been studied extensively for AGN variability
\citep[e.g.][]{2009ApJ...698..895K, 2011ApJ...735...80Z, 2013ApJ...765..106Z}.

For the MCMC we used STAN \citep{nuts} and DNest \citep{dnest}.

\section{Demonstration on Simulated Data}


\section*{Acknowledgements}




\begin{thebibliography}{99}

\bibitem[\protect\citeauthoryear{Bovy Jo, Hogg, 
\& Roweis}{2011}]{extreme_deconvolution} Bovy Jo, Hogg D.~W., Roweis S.~T., 2011, AnApS, 5, 1657 

\bibitem[\protect\citeauthoryear{Brewer, P{\'a}rtay,
\& Cs{\'a}nyi}{2011}]{dnest} Brewer B.~J., P{\'a}rtay L.~B.,
Cs{\'a}nyi G., 2011, Statistics and Computing, 21, 4, 649-656. arXiv:0912.2380

\bibitem[\protect\citeauthoryear{Fine et al.}{2013}]{2013MNRAS.434L..16F} 
Fine S., et al., 2013, MNRAS, 434, L16 

\bibitem[\protect\citeauthoryear{Fine et al.}{2012}]{2012MNRAS.427.2701F} 
Fine S., et al., 2012, MNRAS, 427, 2701 

\bibitem[\protect\citeauthoryear{Kelly}{2007}]{kelly} Kelly 
B.~C., 2007, ApJ, 665, 1489 

\bibitem[\protect\citeauthoryear{Kelly, Bechtold, 
\& Siemiginowska}{2009}]{2009ApJ...698..895K} Kelly B.~C., Bechtold J.,
Siemiginowska A., 2009, ApJ, 698, 895 

\bibitem[\protect\citeauthoryear{Loredo}{2004}]{loredo} Loredo 
T.~J., 2004, AIP Conference Series, 735, 195 

\bibitem[\protect\citeauthoryear{Loredo}{2012}]{2012arXiv1208.3036L} Loredo 
T.~J., 2012, arXiv, arXiv:1208.3036 

\bibitem[\protect\citeauthoryear{Hoffman 
\& Gelman}{2011}]{nuts} Hoffman M.~D., Gelman A., 2011, arXiv, arXiv:1111.4246 

\bibitem[\protect\citeauthoryear{Zu et al.}{2013}]{2013ApJ...765..106Z} Zu 
Y., Kochanek C.~S., Koz{\l}owski S., Udalski A., 2013, ApJ, 765, 106 

\bibitem[\protect\citeauthoryear{Zu, Kochanek, 
\& Peterson}{2011}]{2011ApJ...735...80Z} Zu Y., Kochanek C.~S., Peterson B.~M., 2011, ApJ, 735, 80 



\end{thebibliography}



\end{document}

%%%%%%%%%%%%%%%%%%%%%%%%%%%%%%%%%%%%%%%%%%%%%%%%%%%%%%%%%%%%%%%%%%%%%%%%%%%%%%
